% Chapter 01
\cleardoublepage

\chapter{代数的起源}

\section{简谈代数}

\section{几个典型问题}

\section{线性方程初步}

\section{低阶行列式}

\problem{2}
\begin{quotation}
    证明在三阶行列式展开式中的六项不可能同时为正.
\end{quotation}

\begin{proof}

展开式中的六项的乘积为
\begin{equation}\label{eq:ch01:sec04:prob02}
\begin{aligned}
    \Gamma &= a_{11}a_{22}a_{33}\cdot{}a_{12}a_{23}a_{31}\cdot{a_{13}}a_{21}a_{32}
    \cdot(-a_{11}a_{23}a_{32})\cdot(-a_{12}a_{21}a_{33})\cdot(-a_{13}a_{22}a_{31}) \\
    &= -(a_{11}a_{12}a_{13}a_{21}a_{22}a_{23}a_{31}a_{32}a_{33})^2 \le 0
\end{aligned}
\end{equation}

若这六项同时为正,则应有
\begin{equation}
    \Gamma > 0
\end{equation}

与\abref{eq:ch01:sec04:prob02} 矛盾.

故在三阶行列式展开式中的六项不可能同时为正.

\end{proof}

\problem{3}
\begin{quotation}
验证
\[
\begin{vmatrix}
    a_{11} & a_{12} & a_{13} \\
    a_{21} & a_{22} & a_{23} \\
    a_{31} & a_{32} & a_{33}
\end{vmatrix}
=
\begin{vmatrix}
    a_{11} & a_{21} & a_{31} \\
    a_{12} & a_{22} & a_{32} \\
    a_{13} & a_{23} & a_{33}
\end{vmatrix}
, \quad
\begin{vmatrix}
    0 & a & b \\
    -a & 0 & c \\
    -b & -c & 0
\end{vmatrix}
= 0.
\]
\end{quotation}



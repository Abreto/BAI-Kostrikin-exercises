% Chapter 01
\cleardoublepage

\chapter{代数的起源}

\section{简谈代数}

\section{几个典型问题}

\section{线性方程初步}

\section{低阶行列式}

\problem{2}
\begin{quotation}
    证明在三阶行列式展开式中的六项不可能同时为正.
\end{quotation}

\begin{proof}

展开式中的六项的乘积为
\begin{equation}\label{eq:ch01:sec04:prob02}
\begin{aligned}
    \Gamma &= a_{11}a_{22}a_{33}\cdot{}a_{12}a_{23}a_{31}\cdot{a_{13}}a_{21}a_{32}
    \cdot(-a_{11}a_{23}a_{32})\cdot(-a_{12}a_{21}a_{33})\cdot(-a_{13}a_{22}a_{31}) \\
    &= -(a_{11}a_{12}a_{13}a_{21}a_{22}a_{23}a_{31}a_{32}a_{33})^2 \le 0
\end{aligned}
\end{equation}

若这六项同时为正,则应有
\begin{equation}
    \Gamma > 0
\end{equation}

与\abref{eq:ch01:sec04:prob02} 矛盾.

故在三阶行列式展开式中的六项不可能同时为正.

\end{proof}

\problem{3}
\begin{quotation}
验证
\[
\begin{vmatrix}
    a_{11} & a_{12} & a_{13} \\
    a_{21} & a_{22} & a_{23} \\
    a_{31} & a_{32} & a_{33}
\end{vmatrix}
=
\begin{vmatrix}
    a_{11} & a_{21} & a_{31} \\
    a_{12} & a_{22} & a_{32} \\
    a_{13} & a_{23} & a_{33}
\end{vmatrix}
, \quad
\begin{vmatrix}
    0 & a & b \\
    -a & 0 & c \\
    -b & -c & 0
\end{vmatrix}
= 0.
\]
\end{quotation}

\begin{lemma}\label{lem:ch01:sec04:lem1}
    \[
        \begin{vmatrix}
            a & b \\ c & d
        \end{vmatrix}
        =
        \begin{vmatrix}
            a & c \\ b & d
        \end{vmatrix}
    \]
\end{lemma}
\lemmaref{lem:ch01:sec04:lem1} 可由定义直接导出.

\begin{lemma}\label{lem:ch01:sec04:2}
\[
    -a\begin{vmatrix} c & d \\ e & f \end{vmatrix}
    +b\begin{vmatrix} c & d \\ g & h \end{vmatrix}
    =
    -c\abdet{a & b \\ h & f}
    +d\abdet{a & b \\ g & e}
\]
\end{lemma}
\begin{proof}
\[
\begin{aligned}
    -a\abdet{c & d \\ e & f} + b\abdet{c & d \\ g & h}
    &= -a(cf - de) + b(ch - dg) \\
    &= -c(af - bh) + d(ae - bg) \\
    &= -c\abdet{a & b \\ h & f}+d\abdet{a & b \\ g & e}
\end{aligned}
\]
\end{proof}

\paragraph{第一个式子}

\begin{equation}
\begin{aligned}
    D_1 &= \abdet{a_{11} & a_{12} & a_{13} \\ a_{21} & a_{22} & a_{23} \\ a_{31} & a_{32} & a_{33}} \\
    &= a_{11}\abdet{a_{22} & a_{23} \\ a_{32} & a_{33}}
      -a_{21}\abdet{a_{12} & a_{13} \\ a_{32} & a_{33}}
      +a_{31}\abdet{a_{12} & a_{13} \\ a_{22} & a_{23}}
\end{aligned} 
\end{equation}

由\lemmaref{lem:ch01:sec04:lem1} 可知
\begin{equation}
    a_{11}\abdet{a_{22} & a_{23} \\ a_{32} & a_{33}}
    =
    a_{11}\abdet{a_{22} & a_{32} \\ a_{23} & a_{33}}
\end{equation}

由\lemmaref{lem:ch01:sec04:2} 得
\begin{equation}
    -a_{21}\abdet{a_{12} & a_{13} \\ a_{32} & a_{33}}
    +a_{31}\abdet{a_{12} & a_{13} \\ a_{22} & a_{23}}
    =
    -a_{12}\abdet{a_{21} & a_{31} \\ a_{23} & a_{33}}
    +a_{13}\abdet{a_{21} & a_{31} \\ a_{22} & a_{32}}
\end{equation}

于是
\begin{equation}
\begin{aligned}
    D_1 &= a_{11}\abdet{a_{22} & a_{32} \\ a_{23} & a_{33}}
    -a_{12}\abdet{a_{21} & a_{31} \\ a_{23} & a_{33}}
    +a_{13}\abdet{a_{21} & a_{31} \\ a_{22} & a_{32}} \\
    &= \abdet{
        a_{11} & a_{21} & a_{31} \\
        a_{12} & a_{22} & a_{32} \\
        a_{13} & a_{23} & a_{33}
    }
\end{aligned}
\end{equation}
第一个式子验证完毕.

\paragraph{第二个式子} % (fold)

\begin{equation}
\begin{aligned}
\abdet{
    0 & a & b \\
    -a & 0 & c \\
    -b & -c & 0
} &= 0\abdet{0 & c \\ -c & 0}
    -(-a)\abdet{a & b \\ -c & 0}
    +(-b)\abdet{a & b \\ 0 & c} \\
  &= abc - bac = 0
\end{aligned}
\end{equation}

验证完毕.

% paragraph 第二个式子 (end)

\section{集合与映射} % (begin)

\begin{lemma}\label{lem:ch01:sec05:1}
    设
    \[
        f: X \to Y, \qquad g: Y \to X
    \]
    是任意两个映射,如果 $gf = e_X$,则 $f$ 是单的,$g$ 是满的.
\end{lemma}

\problem{1}
\begin{quotation}
    设 $\Omega = \{+,-,++,+-,-+,--,+++,\cdots\}$ 是加号和减号的有限序列的集合,
    而 $f: \Omega \to \Omega$ 是一个变换,将元素
    $\omega = \omega_1\omega_2\cdots\omega_n\in\Omega$
    对应到
    $\omega' = \omega_1\dot{\omega}_1\omega_2\dot{\omega}_2\cdots\omega_n\dot{\omega}_n$,
    其中若 $\omega_k=+$,则 $\dot{\omega}_k=-$,
    若 $\omega_k=-$,则 $\dot{\omega}_k=+$.
    证明在 $f(f\omega)$ 的长度 $>4$ 的任意区间内包含 $++$ 或 $--$.
\end{quotation}

\begin{proof}
    设
    \begin{equation}
        f(f\omega) = \omega_0\omega_1\omega_2\cdots\omega_n
    \end{equation}

    由定义易知,
    \begin{equation}
        \forall k \ge 0, \omega_{4k+1} = \omega_{4k+2}
    \end{equation}

    对于任意长度 $>4$ 的区间,一定包含连续的两项形如 $\omega_{4k+1}\omega_{4k+2}$.

    即一定包含相等的连续两项.
\end{proof}

% section 集合与映射 (end)
